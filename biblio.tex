% La bibliografia, da inserirsi solo se ci sono state citazioni.
% In questo caso ricordarsi che bisogna sempre elaborare due volte il file .TEX
% perché la prima volta viene generata la bibliografia mentre la seconda volta viene inclusa

% NOTA: citare il DOI non è obbligatorio ma MOLTO desiderabile

%\begin{thebibliography}{9} % se ci sono meno di 10 citazioni
\begin{thebibliography}{99} % se ci sono da 10 a 99 citazioni

\bibitem{stopbadawareSurvey}
Stop BadAware Survey for Compromised Websites
\url{http://www.stopbadware.org/files/compromised-websites-an-owners-perspective.pdf}

\bibitem{honeynetProject}
The Honeynet project, % nome del progetto
\url{http://www.honeynet.org/} % URI della pagina web

\bibitem{leurre}
F.Pouget, M.Dacier, V.H.Pham
``Leurre.com: on the advantages of deploying a large scale distributed honeypot platform''
ECCE 2005, E-Crime and Computer Conference, March 29-30 2005,
Monaco, France
pp.\ 29-30

\bibitem{sgnet}
C.Leita, M.Dacier,
``Sgnet: A worldwide deployable framework to support the analysis of malware threat models''
EDCC 2008, 7th European Dependable Computing Conference,
Kaunas, Lituania, May 7-9, 2008
\doi{10.1109/EDCC-7.2008.15}

\bibitem{honeyd}
N. Provos,
``A virtual honeypot framework'',
USENIX Security Symposium,
San Diego (California - USA), August 9-13, 2004,
pp.\ 1-14

\bibitem{highhoney}
V. Nicomette, M. Kaaniche, E. Alata, and M. Herrb,
``Set-up and Deployment of a High Interaction Honeypot: Experiment and Lessons Learned'',
Journal in Computer Virology,
vol.\ 7, no.\ 2,
May 2011,
pp.\ 143-157,
\doi{10.1007/s11416-010-0144-2}

\bibitem{googleHoney}
Google Hack Honeypot,
\url{http://ghh.sourceforge.net}

\bibitem{dswhp}
DShield web honeypot project,
\url{https://sites.google.com/site/webhoneypotsite/}

\bibitem{spitzhoney}
To Build A Honeypot, Lance Spitzner,
\url{http://www.spitzner.net/honeypot.html}

\bibitem{glastopf}
Glastopf project,
\url{http://honeynet.org/files/KYT-Glastopf-Final_v1.pdf}

\bibitem{johnhsh}
J. P. John, F. Yu, Y. Xie, A. Krishnamurthy, M. Abadi,
``Heat-seeking honeypots: design and experience'',
International World Wide Web Conference (WWW),
New York (NY - USA), March-April 2011,
pp.\ 207-216,
\doi{10.1145/1963405.1963437}

\bibitem{hihat}
M. Muüter, F. Freiling, T. Holz, and J. Matthews,
``A generic toolkit for converting web applications into high-interaction honeypots''
Recent Advances in Intrusion Detection (RAID),
pp.\ 154-170,
Gold Coast (Australia), September 5-7, 2007.

\bibitem{sshprofiling}
D.Ramsbrock, R.Berthier, and M.Cukier,
``Profiling attacker behaviour following ssh compromises''
IEEE/IFIP International Conference on Dependable Systems and Networks, 2007.
\doi{10.1109/DSN.2007.76}

\bibitem{plagdet1}
X. Chen, B. Francia, M. Li, B. Mckinnon, and A. Seker,
``Shared information and program plagiarism detection''
IEEE Transactions on Information Theory,
Vol.\ 50, No.\ 7,
July 2004,
pp.\ 1545-1551,
\doi{10.1109/TIT.2004.830793}

\bibitem{plagdet2}
A. Saebjornsen, J. Willcock, T. Panas, D. Quinlan, and Z. Su,
``Detecting code clones in binary executables'',
International Symposium on Software testing and analysis, ISSTA 2009,
 Chicago (Illinois - USA), July 19-23, 2009,
pp.\ 117-128,
\doi{10.1145/1572272.1572287}

\bibitem{ssdeep}
J. Kornblum,
``Identifying almost identical files using context triggered piecewise hashing'',
Digital Investigation,
Vol. 3, Supplement(0)
2006.
pp.\ 91-97,
\doi{:10.1016/j.diin.2006.06.015}

\bibitem{sdhash}
V. Roussev,
``Data fingerprinting with similarity digests''
Advances in Digital Forensics VI, volume 337,
Springer Boston, 2010,
pp.\ 207-226,
\doi{10.1007/978-3-642-15506-2_15}

\bibitem{vmescape}
VMWare escape, % nome del progetto
\url{http://www.coresecurity.com/content/advisory-vmware} % URI della pagina web

\bibitem{phpMyAdmin}
phpMyAdmin reference page,
\url{http://www.phpmyadmin.net}

\bibitem{osCommerce}
osCommerce reference page,
\url{http://www.oscommerce.com}

\bibitem{joomla}
Joomla! reference page,
\url{http://www.joomla.org}

\bibitem{wordpress}
Wordpress reference page,
\url{http://www.wordpress.com}

\bibitem{smf}
SMF reference page,
\url{http://www.simplemachines.org}

\bibitem{drupal}
Drupal reference page,
\url{http://www.drupal.org}

\bibitem{conntrack}
Conntrack reference page
\url{http://www.netfilter.org/projects/libnetfilter_conntrack/index.html}

\bibitem{libmagic}
LibMagic Debian package page
\url{http://packages.debian.org/unstable/libdevel/libmagic-dev}

\bibitem{phpcrypt}
php-crypt home page
\url{http://www.php-crypt.com}

\bibitem{webdeobf}
PHP deobfuscation web service
\url{https://www.whitefirdesign.com/tools/deobfuscate-php-hack-code.html}

\bibitem{evalhook}
EvalHook PHP extension, by Stefan Esser
\url{http://php-security.org/2010/05/13/article-decoding-a-user-space-encoded-php-script}

\bibitem{spamsum}
SpamSum spam detection system, first example of context-triggered hash
\url{https://www.samba.org/ftp/unpacked/junkcode/spamsum/}

\bibitem{md5}
MD5 RFC 1321
\url{http://www.ietf.org/rfc/rfc1321.txt}

\bibitem{node_home}
Node.Js homepage
\url{http://nodejs.org/}

\bibitem{express_node}
Express framework homepage
\url{http://expressjs.com/}

\bibitem{d3_home}
d3 homepage
\url{http://d3js.org/}

\bibitem{bootstrap}
Bootstrap homepage
\url{http://getbootstrap.com/2.3.2/}

\bibitem{devilfinder}
Devilfinder.com, search engine
\url{http://devilfinder.com}

\bibitem{googleSafeBrowsing}
Google Safe Browsing website
\url{http://www.google.com/transparencyreport/safebrowsing/}

\bibitem{wepaWet}
Wepawet home page
\url{https://wepawet.iseclab.org/}

\bibitem{jdgui}
JD-GUI home page
\url{http://jd.benow.ca/}

\bibitem{cveJava}
CVE-2013-0422 <Java7.11 RCE outside SandBox
\url{http://cvedetails.com/cve/2013-0422}

\bibitem{CVE-2010-0188}
CVE-2010-0188 Adobe Acrobat <8.21 and <9.21 RCE
\url{http://cvedetails.com/cve/2010-0188}

\bibitem{CVE-2010-1885}
CVE 2010-1885 MS Win XP Help Center URL Validation RCE
\url{http://cvedetails.com/cve/2010-1885}

\bibitem{nessus}
Nessus network scanner
\url{http://www.tenable.com/products/nessus}

\bibitem{nmap}
Nmap port mapper
\url{http://nmap.org/}

\bibitem{malwWin}
G-Data report For Malware OS percentages
\url{http://www.gdatasoftware.co.uk/press-center/news/article/article/1760-number-of-new-computer-viruses.html}

\bibitem{virustotal}
VirusTotal Home Page
\url{https://www.virustotal.com/}

\bibitem{joomlaImgManager}
Joomla! img\_manager Arbitrary File Upload vulnerability
\url{http://www.exploit-db.com/exploits/17734/}

\bibitem{irc}
IRC RFC 1459
\url{http://tools.ietf.org/html/rfc1459.html}

\bibitem{zoneh}
Zone-h home page
\url{https://www.zone-h.org/}

\bibitem{python}
Python home page
\url{https://www.python.org/}

\bibitem{perl}
Perl home page
\url{https://www.perl.org/}

\bibitem{php}
PHP home page
\url{https://www.php.net/}

\bibitem{curl}
cURL home page
\url{https://www.curl.haxx.se/}

\bibitem{apache}
Apache home page
\url{https://www.apache.org/}

\bibitem{mysql}
MySQL home page
\url{https://www.mysql.com/}

\bibitem{echolot}
Echolot home page
\url{http://www.palfrader.org/code/echolot/}

\end{thebibliography}
