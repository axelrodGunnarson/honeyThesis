\section{Conclusions}

In this research we described the implementation and deployment of a honeypot network based on a number of real, vulnerable web applications. Using the collected data, we studied the behavior of the attackers before, during, and after they compromise their targets. Our experiments run successfully for more than 100 days, collecting several GB of data and successfully identifying the most common trends in nowadays web attacks.

The results of our study provide interesting insights on the current state of exploitation behaviors on the web. On one side, we were able to confirm known trends for certain classes of attacks, such as the prevalence of eastern European countries in comment spamming activities, and the fact that many of the scam and phishing campaigns are still operated by criminals in African countries. Pharmaceutical ads appear to be the most common subject among spam and comment spamming activities, as found by other recent studies.

On the other hand, we were also able to observe and study a large number of manual attacks, as well as many infections aimed at turning web servers into IRC bots. This suggests that some of the threats that are often considered outdated are actually still very popular (in particular between young criminals) and are still responsible for a large fraction of the attacks against vulnerable websites. Defacement is another activity which, while not being really dangerous in terms of client-security, can be very annoying for webmasters, who have to restore the machine to a past state. Furthermore, the lack of legal jurisdiction about malicious activities performed on the Internet works as an attractor for computer-savvy teenagers.

Furthermore, we created a platform which can be easily replicated and which is very scalable. Adding new applications, as well as adding new domains, is a low time-consuming task, allowing for fast updates and improvements of the system.

\section{Future Works}

Our honeypot platform is currently running and receiving requests. However, after more than 100 days we begin to notice a slight decrease in the number of attacks performed to our servers. We believe that some attackers are starting to understand that our domains are connecting to a honeypot and not a real web server, and they are blacklisting our IPs from their scans. We are going to tackle this problem by changing both domains and IPs of the proxy (our modular infrastructure allows to easily perform this sort of tasks, without changing the whole platform).

Furthermore, we are going to replace some of the web applications we used in order to be updated with new vulnerabilities, as we noticed that over time we received less and less attacks on the oldest applications (Joomla!). Because each application is hosted on its own virtual web server, adding a new application is as easy as inserting a new virtual machine image to the VM server.

Finally, we collected several GB of data. We can always perform more analysis on the different files, as exploring the development of received files, looking at which functionalities are implemented in newer versions of web shells or which updates bot scripts are receiving from C\&C servers. Other than files, we also collected all requests attackers performed from our machines toward the external world. We can analyze these requests, analyzing URLs and domains, in order to understand the source attackers use for downloading new files, and eventually collaborate with web-hosting providers in order to make automatic download of files a more difficult task.
